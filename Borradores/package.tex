
\usepackage[spanish, mexico]{babel}
\usepackage[utf8]{inputenc}
\usepackage[T1]{fontenc} 
\usepackage{makeidx}
\usepackage[pdftex]{graphicx} %Para incluir figuras, sin reparar en la extensión
\usepackage{amsmath, amssymb, amsfonts, latexsym, mathrsfs}
\usepackage{amsthm}
\newtheorem*{definition}{Definition}
\usepackage{cancel}
\usepackage{pifont}
\usepackage[pdftex, colorlinks = true, linkcolor = blue]{hyperref}
\usepackage[left = 2cm, right = 2cm, top = 2cm, bottom = 2cm]{geometry}
\usepackage{mdwlist}
\usepackage{multirow}
\usepackage{rotating}
\usepackage{lscape}
\usepackage{longtable}
\usepackage{verbatim, fancyvrb}
\usepackage{units}
\usepackage{pgf, tikz}
\usetikzlibrary{arrows}
\usepackage{colortbl}
\definecolor{magenta4}{rgb}{0.55,0,0.55}

\usepackage[margin=3cm, font=footnotesize, labelfont=bf]{caption}
\usepackage{float}
\usepackage{subfigure}


\setlength{\parindent}{0cm} % Sin Sangría

\usepackage{paralist} % Para enumera automaticamente \begin{enumerate}[(a)]

\usepackage{fancyhdr} % Inserta en el encabeza el numero del capitulo y una linea horizontal.
\pagestyle{fancy}
%\fancyhead[LO]{\leftmark} % En la parte izquierda del encabezado, aparecerá el nombre de capítulo
\fancyhead[RO,LE]{\thepage} % Números de página en las esquinas de los encabezados
%\usepackage{bbding}
%\usepackage{epstopdf} % Convertir .eps a .pdf (si fuera necesario)
%\DeclareGraphicsExtensions{.pdf,.png,.jpg} % busca en este orden

%\usepackage[dvips]{graphicx}        % standard LaTeX graphics tool
                             % when including figure files
%\usepackage{multicol}        % used for the two-column index
%\usepackage[bottom]{footmisc}% places footnotes at page bottom
%\usepackage{type1cm}   
%\usepackage{fancyhdr}
%\usepackage{mathpazo}

%\addto{\captionsspanish}{\def\chaptername{}}
%\addto{\captionsspanish}{\def\thechapter{}}


\parskip=1mm % Para separar textos

%\title{}
%\author{}
%\date{}

\theoremstyle{plain}
\newtheorem{theorem}{Teorema}[section]{\bfseries}{\scshape}  
\newtheorem{coro}{Corolario}[section]{\bfseries}{\itshape} 
\newtheorem{propo}{Proposici\'on}[section]{\bfseries}{\itshape} 
\newtheorem{lemma}{Lema}[section]{\bfseries}{\itshape}
\newtheorem{defi}{Definici\'on}[section]{\bfseries}{\itshape} 

\theoremstyle{definition}
\newtheorem{exercise}{Ejercicio}{\bfseries}{\rmfamily}
\newtheorem*{solution}{Soluci\'on}{\bfseries}{\rmfamily}
\newtheorem{exam}{Ejemplo}{\bfseries}{\rmfamily}

%\newtheorem*{demo}{Demostraci\'on}{\bfseries}{\rmfamily}
\newtheorem*{remark}{Observaci\'on}{\bfseries}{\rmfamily}


% end of proof symbol
%\newcommand\qedsymbol{\hbox{\rlap{$\sqcap$}$\sqcup$}}
%\newcommand\qed{\relax\ifmmode\else\unskip\quad\fi\qedsymbol}
%\newcommand\smartqed{\renewcommand\qed{\relax\ifmmode\qedsymbol\else
%  {\unskip\nobreak\hfil\penalty50\hskip1em\null\nobreak\hfil\qedsymbol
%  \parfillskip=\z@\finalhyphendemerits=0\endgraf}\fi}}

% LO QUE YO HE DEFINIDO

\newcommand{\cur}[1]{{\em #1}}
\newcommand{\der}[2]{\dfrac{\mbox{d} #1}{\mbox{d} #2}}
\newcommand{\partione}[2]{\partial #1 / \partial #2}
\newcommand{\partitwo}[2]{\dfrac{\partial #1}{\partial #2}}
\newcommand{\inteuno}[2]{\displaystyle \int \, #1 \, \mbox{d}#2}
\newcommand{\intedos}[4]{\displaystyle \int_{#1}^{#2} \, #3 \, \mbox{d}#4}
\newcommand{\negri}[1]{{\bf #1}}
\newcommand{\pvinicial}[2]{(#1 _0, #2 _0)}
\newcommand{\inter}{(t_0 - h, t_0 + h)}
\newcommand{\maxi}{(\alpha, \omega)}
\newcommand{\pvi}{{\rm PVI} }
\newcommand{\e}{{e}}
\newcommand{\Dconj}{\mathscr{D}}
\newcommand{\nbr}{\mathcal{R}_0}
\newcommand{\clas}{\mathscr{C}}
\newcommand{\enf}[1]{\guillemotleft #1\guillemotright}
\newcommand{\prom}{\bar{x}}
\newcommand{\com}[1]{$-$ #1 $-$}
\newcommand{\matmod}[1]{\begin{align}  #1  \end{align}}
\newcommand{\solpa}{x(t; t_0, x_0)}
\newcommand{\solu}{\varphi(t; t_0, x_0)}
\newcommand{\bart}{\bar{t}}
\newcommand{\solua}{\varphi(\bart; \bart_0, \prom_0)}
\newcommand{\solub}{\varphi(\bart; \bart_0, x_0)}
\newcommand{\soluc}{\varphi(t; \bart_0, \prom_0)}




% Algunos símbolos matemáticos
\newcommand{\R}{\mathbb{R}}
\newcommand{\C}{\mathbb{C}}
\newcommand{\N}{\mathbb{N}}
\newcommand{\Z}{\mathbb{Z}}


\renewcommand{\qedsymbol}{$\blacksquare$}

%\renewcommand{\baselinestretch}{1.2}

%\renewcommand{\refname}{BIBLIOGRAF\'IA}

% TERMINAN MIS DEFINICIONES


% dedication environment
%\newenvironment{dedication}
%{\clearemptydoublepage
%\thispagestyle{empty}
%\vspace*{13\baselineskip}
%\large\itshape
%\let\\\@centercr\@rightskip\@flushglue \rightskip\@rightskip
%\leftskip4cm\parindent\z@\relax
%\everypar{\parindent=\svparindent\let\everypar\empty}}{\clearpage}
%
% predefined unnumbered headings
\newcommand{\preface}[1][\prefacename]{\chapter*{#1}\markboth{#1}{#1}}
\newcommand{\foreword}[1][\forewordname]{\chapter*{#1}\markboth{#1}{#1}}






