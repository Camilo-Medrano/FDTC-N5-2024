%\documentclass[letterpaper, 12pt]{article}

%Paquetes a utilizarse
%
\usepackage[spanish, mexico]{babel}
\usepackage[utf8]{inputenc}
\usepackage[T1]{fontenc} 
\usepackage{makeidx}
\usepackage[pdftex]{graphicx} %Para incluir figuras, sin reparar en la extensión
\usepackage{amsmath, amssymb, amsfonts, latexsym, mathrsfs}
\usepackage{amsthm}
\newtheorem*{definition}{Definition}
\usepackage{cancel}
\usepackage{pifont}
\usepackage[pdftex, colorlinks = true, linkcolor = blue]{hyperref}
\usepackage[left = 2cm, right = 2cm, top = 2cm, bottom = 2cm]{geometry}
\usepackage{mdwlist}
\usepackage{multirow}
\usepackage{rotating}
\usepackage{lscape}
\usepackage{longtable}
\usepackage{verbatim, fancyvrb}
\usepackage{units}
\usepackage{pgf, tikz}
\usetikzlibrary{arrows}
\usepackage{colortbl}
\definecolor{magenta4}{rgb}{0.55,0,0.55}

\usepackage[margin=3cm, font=footnotesize, labelfont=bf]{caption}
\usepackage{float}
\usepackage{subfigure}


\setlength{\parindent}{0cm} % Sin Sangría

\usepackage{paralist} % Para enumera automaticamente \begin{enumerate}[(a)]

\usepackage{fancyhdr} % Inserta en el encabeza el numero del capitulo y una linea horizontal.
\pagestyle{fancy}
%\fancyhead[LO]{\leftmark} % En la parte izquierda del encabezado, aparecerá el nombre de capítulo
\fancyhead[RO,LE]{\thepage} % Números de página en las esquinas de los encabezados
%\usepackage{bbding}
%\usepackage{epstopdf} % Convertir .eps a .pdf (si fuera necesario)
%\DeclareGraphicsExtensions{.pdf,.png,.jpg} % busca en este orden

%\usepackage[dvips]{graphicx}        % standard LaTeX graphics tool
                             % when including figure files
%\usepackage{multicol}        % used for the two-column index
%\usepackage[bottom]{footmisc}% places footnotes at page bottom
%\usepackage{type1cm}   
%\usepackage{fancyhdr}
%\usepackage{mathpazo}

%\addto{\captionsspanish}{\def\chaptername{}}
%\addto{\captionsspanish}{\def\thechapter{}}


\parskip=1mm % Para separar textos

%\title{}
%\author{}
%\date{}

\theoremstyle{plain}
\newtheorem{theorem}{Teorema}[section]{\bfseries}{\scshape}  
\newtheorem{coro}{Corolario}[section]{\bfseries}{\itshape} 
\newtheorem{propo}{Proposici\'on}[section]{\bfseries}{\itshape} 
\newtheorem{lemma}{Lema}[section]{\bfseries}{\itshape}
\newtheorem{defi}{Definici\'on}[section]{\bfseries}{\itshape} 

\theoremstyle{definition}
\newtheorem{exercise}{Ejercicio}{\bfseries}{\rmfamily}
\newtheorem*{solution}{Soluci\'on}{\bfseries}{\rmfamily}
\newtheorem{exam}{Ejemplo}{\bfseries}{\rmfamily}

%\newtheorem*{demo}{Demostraci\'on}{\bfseries}{\rmfamily}
\newtheorem*{remark}{Observaci\'on}{\bfseries}{\rmfamily}


% end of proof symbol
%\newcommand\qedsymbol{\hbox{\rlap{$\sqcap$}$\sqcup$}}
%\newcommand\qed{\relax\ifmmode\else\unskip\quad\fi\qedsymbol}
%\newcommand\smartqed{\renewcommand\qed{\relax\ifmmode\qedsymbol\else
%  {\unskip\nobreak\hfil\penalty50\hskip1em\null\nobreak\hfil\qedsymbol
%  \parfillskip=\z@\finalhyphendemerits=0\endgraf}\fi}}

% LO QUE YO HE DEFINIDO

\newcommand{\cur}[1]{{\em #1}}
\newcommand{\der}[2]{\dfrac{\mbox{d} #1}{\mbox{d} #2}}
\newcommand{\partione}[2]{\partial #1 / \partial #2}
\newcommand{\partitwo}[2]{\dfrac{\partial #1}{\partial #2}}
\newcommand{\inteuno}[2]{\displaystyle \int \, #1 \, \mbox{d}#2}
\newcommand{\intedos}[4]{\displaystyle \int_{#1}^{#2} \, #3 \, \mbox{d}#4}
\newcommand{\negri}[1]{{\bf #1}}
\newcommand{\pvinicial}[2]{(#1 _0, #2 _0)}
\newcommand{\inter}{(t_0 - h, t_0 + h)}
\newcommand{\maxi}{(\alpha, \omega)}
\newcommand{\pvi}{{\rm PVI} }
\newcommand{\e}{{e}}
\newcommand{\Dconj}{\mathscr{D}}
\newcommand{\nbr}{\mathcal{R}_0}
\newcommand{\clas}{\mathscr{C}}
\newcommand{\enf}[1]{\guillemotleft #1\guillemotright}
\newcommand{\prom}{\bar{x}}
\newcommand{\com}[1]{$-$ #1 $-$}
\newcommand{\matmod}[1]{\begin{align}  #1  \end{align}}
\newcommand{\solpa}{x(t; t_0, x_0)}
\newcommand{\solu}{\varphi(t; t_0, x_0)}
\newcommand{\bart}{\bar{t}}
\newcommand{\solua}{\varphi(\bart; \bart_0, \prom_0)}
\newcommand{\solub}{\varphi(\bart; \bart_0, x_0)}
\newcommand{\soluc}{\varphi(t; \bart_0, \prom_0)}




% Algunos símbolos matemáticos
\newcommand{\R}{\mathbb{R}}
\newcommand{\C}{\mathbb{C}}
\newcommand{\N}{\mathbb{N}}
\newcommand{\Z}{\mathbb{Z}}


\renewcommand{\qedsymbol}{$\blacksquare$}

%\renewcommand{\baselinestretch}{1.2}

%\renewcommand{\refname}{BIBLIOGRAF\'IA}

% TERMINAN MIS DEFINICIONES


% dedication environment
%\newenvironment{dedication}
%{\clearemptydoublepage
%\thispagestyle{empty}
%\vspace*{13\baselineskip}
%\large\itshape
%\let\\\@centercr\@rightskip\@flushglue \rightskip\@rightskip
%\leftskip4cm\parindent\z@\relax
%\everypar{\parindent=\svparindent\let\everypar\empty}}{\clearpage}
%
% predefined unnumbered headings
\newcommand{\preface}[1][\prefacename]{\chapter*{#1}\markboth{#1}{#1}}
\newcommand{\foreword}[1][\forewordname]{\chapter*{#1}\markboth{#1}{#1}}

\newtheorem{obs}{Observación}






%\usepackage[margin=2.5cm]{geometry}
%\usepackage{wasysym}
%\newcommand{\fecha}{Agosto, 2014}
%\usepackage{stmaryrd,textcomp}
%\usepackage{pgf,tikz}
%\usetikzlibrary{arrows}

\parskip = 2mm   %%%% genera un espacio de X mm entre lo párrafos
\parindent = 3mm
%\usepackage{iwona}

\definecolor{yellowgreen}{rgb}{0.6,0.8,0.2}


\thispagestyle{empty}
%\begin{center}
%\begin{tabular}{l b{12cm} r}
%\includegraphics[width=3.5cm,height=4cm]{UES.png}
%		\hspace{3cm}\includegraphics[width=6cm]{logomined.jpg}
%		&
%		\hspace{2cm} \includegraphics[height=4cm]{PJT.jpg}
%		& 
%\end{tabular}
%\end{center}
\pagestyle{plain}
\begin{center}
\begin{flushleft}\hspace{55mm}
\includegraphics[height=2.7cm]{SILABO/LOGO 2020 MINED.png}
\end{flushleft}

\,\\

UNIVERSIDAD DE EL SALVADOR\\
    PROGRAMA JÓVENES TALENTO\\
    MINISTERIO DE EDUCACIÓN
\end{center}
%\begin{flushleft}
%Facultad de Ciencias Naturales y Matem\'aticas\hfill \\
%Programa Jóvenes Talento\\
%\underline{Universidad de El Salvador}
%\end{flushleft}

\begin{flushleft}\vspace{-45mm}
\includegraphics[height=3.6cm]{SILABO/LOGO UES 2020.png}
\end{flushleft}

 
\begin{flushright}\vspace{-40mm}
\includegraphics[height=2cm]{SILABO/logo.png}
\end{flushright}
\,\\
\,\\

%\begin{center}
% {\bf
%UNIVERSIDAD DE EL SALVADOR\\ [0.3cm]
%MINISTERIO DE EDUCACI\'ON \\ [0.3cm]
%PROGRAMA JÓVENES TALENTO-MINISTERIO DE EDUCACIÓN }  \\ [1.3cm]
%\end{center}

\begin{center}

{\bf \Large SÍLABO - NIVEL V}\\  
\vspace{0.3ex}%
\rule{\textwidth}{2pt}
{\Huge \bf COMBINATORIA\\  
\vspace{0.7ex}%
\rule{\textwidth}{2pt}
 }
\end{center}

\vspace{0.3cm}

\begin{center}
%\includegraphics[scale=0.6]{Logo8.jpg}
\end{center}

\vspace{0.3cm}

\begin{center}
{\bf SAN SALVADOR, EL SALVADOR, DICIEMBRE DE 2024}
\end{center}








\newpage

\begin{enumerate}
\thispagestyle{empty}

\item  {\bf \large  I. GENERALIDADES.}

\vspace{1cm}
{\bf EQUIPO DE TRABAJO:}


\textbf{MENTOR:}\\
 M.Sc. Kevyn Jaime Murcia Mayorga.\\
%T\'itulo: Maestro en matemática fundamental.\\
\textbf{Correo:} kevin.murcia@jovenestalento.edu.sv\\

\textbf{INSTRUCTORES:}\\
Pedro Julio Avelar Hernández\\
%Nivel educativo: Profesorado en Matem\'atica\\
\textbf{Correo:} pedro.avelar@jovenestalento.edu.sv\\

Camilo Samuel Medrano Martínez\\
\textbf{Correo:} camilo.medrano@jovenestalento.edu.sv\\


\item {\bf \large II. DESCRIPCI\'ON DEL CURSO.}\\

El curso está orientado a desarrollar la habilidad de resolver problemas en el área de conteo o
que involucren el análisis de configuraciones. Se introduce el estudio de elementos básicos de
la teoría de conjuntos como una herramienta para el desarrollo de contenidos futuros durante
el curso. Se estudiaran elementos del  Análisis combinatorio, los cuales estarán enfocados
en fortalecer las técnicas básicas de conteo y en torno a ellas se formularán las definiciones
y los principios básicos, es decir, el principio de la suma, el producto y correspondencia, para
luego trabajar las combinaciones con tres enfoques: modelo conjuntista, de caminos y de cadenas
binarias. Se completarán luego las técnicas de conteo al estudiar formalmente las permutaciones. Los temas
a tratar, forman parte fundamental para la comprensión tanto en
matemática como en otras ciencias. Por ejemplo, en estadística, el cálculo de probabilidades llega
a ser un pilar fundamental en donde se utilizan ampliamente los teoremas que se estudirán, como
los son el teorema del binomio y el principio de inclusión. Para terminar el curso se estudiará
Comparaciones, Separadores, Combinaciones con grupos de objetos idénticos y el principio de las casillas.




\item {\bf \large  III. OBJETIVOS DEL CURSO.}

\begin{itemize}
\item Desarrollar en el estudiante el pensamiento matemático para la solución de los problemas relacionados con técnicas de conteo.
\item  Aprender y desarrollar las destrazas básicas para la resolución de problemas de conteo. 
\item Aplicar las principales ideas del curso en diferentes áreas de la matemática y otras ciencias
\end{itemize}

\newpage 
\item {\bf \large IV. CONTENIDO A DESARROLLAR.}

\begin{enumerate}
\item Introducci\'on.
    
\begin{enumerate}[ ]
\item [1.1] Introducción
\item [1.2]  Teoría básica de conjuntos.
\end{enumerate}
   \item  Principios de conteo.
   \begin{enumerate}[ ]
       \item [2.1] Principio de la suma
       \item [2.2] Principio de la multiplicaci\'on
       \item [2.3] Principio de correspondencia
       \item [2.4] Recurrencia
       \item [2.5] Principio de inclusión-exclusión
   \end{enumerate}

   
   \item Combinaciones.
   \begin{enumerate}
       \item [3.1] Modelo de conjuntos.
       \item [3.2] Caminos.
       \item [3.3] Modelo de cadenas binarias
   \end{enumerate}

    \item Permutaciones.
   \begin{enumerate}
       \item [4.1] Introducci\'on, notaci\'on y ejemplos.
       \item [4.2] Arreglos.
       \item [4.3] Permutaciones de objetos en una circunferencia.
       \item [4.4] Permutaciones de objetos id\'enticos
       
   \end{enumerate}
  
   \item Teorema del binomio y tri\'angulo de Pascal

   \item Comparaciones (contar de dos formas).
   \item Principio de Inclusión-exclusión. 
   \item Desórdenes
\item Separadores y combinaciones con grupos de objetos idénticos
\item  Principio de casillas
   
    
  
\end{enumerate}





\item {\bf \large VI. METODOLOG\'IA.}\\

Las actividades se diseñaran de tal forma de guiar a los estudiantes hacia el descubrimiento de los conceptos y algoritmos necesarios haciendo énfasis en el desarrollo intuitivo de las definiciones, como paso previo, para luego construir herramientas que permitan la solución de problemas. Se crearán situaciones que incentiven la creatividad e interés de los alumnos. Con el objetivo que el
alumno aplique los conocimientos adquiridos se desarrollaran sesiones de resolución de problemas los martes y jueves de cada semana, en las cuales los alumnos podrán socializar sus soluciones y colocar de manifiesto estrategias y creatividad en las mismas. En esta etapa se podrán visualizar los aspectos que se le dificultan a los alumnos para posteriormente reforzar los temas que pudieran
presentar más dificultad. Se pretende que el estudiante adopte una actitud participativa, y de cooperación con sus compañeros, para ello se brindara la confianza necesaria y el trabajo en conjunto. El curso se desarrollará mediante las siguientes actividades:

\begin{enumerate}
    \item Clases expositiva: las clases serán desarrolladas por los educadores modelo;
especialmente, se orientará al estudiante por medio de guías que desarrollará por su cuenta,
con la finalidad de descubrir y determinar conceptos, fórmulas y propiedades por sí mismos.
\item  Discusión de problemas: se desarrollarán en un espacio de una hora despues del primer bloque de clase. Se buscará que los estudiantes desarrollen, o refuercen sus habilidades para la resolución de problemas con mayor dificultad. 
\item  Tareas: se asignará una tarea diaria para resolver en el aula, luego los instructores modelo resolverán o derán las ideas principales para la solución de esta.
\item  Evaluaciones cortas: se desarrollará exámenes cortos de lo que se vio el día anterior, por lo que los exámenes cortos serán todos los días a excepción del día lunes.
\item Examen semanal: evaluación que se realizará el sábado de cada semana e incluye el contenido estudiado a lo largo de toda semana.

\end{enumerate}

\item  {\bf\large  VI.  EVALUACI\'ON.}
\begin{itemize}
\item \textbf{ Cortos:}\\ Ex\'amenes breves donde se evaluar\'a el material visto en la clase anterior. Su
duración m\'axima será de 30 minutos.\\

\item \textbf{ Tareas:}\\
 Cada día quedará un espacio de 45 minutos para que estudiante pueda desarrollar algunos problemas como tarea, después de la entrega de la tarea los instructores resolverán o darán las ideas principales de la solución.\\

\item \textbf{ Examen semanal:} un examen que contempla el material cubierto durante la semana.\\

\item \textbf{ Cr\'editos Extra:}\\
 Son problemas de desaf\'io,  que se pueden desarrollar utilizando
los conceptos vistos en clase,  pero exigen creatividad de parte de los alumnos dado su nivel
de complejidad. Si un alumno opta no hacerlos siempre puede llegar a 10 como nota, pero
si opta hacerlos puede llegar a tener 11 de nota.

\end{itemize}
Los porcentajes de cada una de las actividades se  detallan a continuaci\'on:

\begin{center}
\begin{table}[h!]
    \centering
    \begin{tabular}{ |p{5cm}||p{5cm}| }
\hline
\textbf{ACTIVIDAD} & \textbf{PONDERACIÓN}\\
\hline
Examénes cortos &  30\%  \\
Tareas &  30\%   \\
Examen semanal    &  40\%   \\
Créditos  &  10\%  \\ 
\hline
\textbf{TOTAL} & 110\%\\
\hline
\end{tabular}
\end{table}

\end{center}
%\begin{center}
  
 %\begin{tabular}{} p{2.5cm}    !{\color{yellowgreen}\vrule}  c   !{\color{yellow}\vrule}} \arrayrulecolor{yellow}  \hline
%\rowcolor{yellow}    \multicolumn{1}{|c}{\textbf{ACTIVIDAD}} & \multicolumn{1}{|c|}{\textbf{PONDERACI\'ON}} \\  \arrayrulecolor{yellowgreen} \hline
%    Cortos &  30\%  \\  \arrayrulecolor{yellowgreen}   \hline
%    Tareas &  30\%        \\  \arrayrulecolor{yellowgreen}     \hline
%   Semanal    &  30\%   \\   \arrayrulecolor{yellowgreen}  	\hline
%    Asistencia    &  10\%   \\   \arrayrulecolor{yellowgreen}  	\hline
%Cr\'editos  &  10\%  \\   \arrayrulecolor{yellowgreen}  \hline
%\textbf{TOTAL}  &  110\%  \\
%		\hline
%\end{tabular}

%\end{center}
\newpage

\item {\bf \large VII. JORNALIZACI\'ON.}

\begin{center}
\begin{table}[h!]
    \centering
    \begin{tabular}{ |p{5cm}||p{5cm}| }
\hline
\textbf{HORA} & \textbf{ACTIVIDAD}\\
\hline
1:30 pm -3:00 pm &  Clase Teórica   \\
3:00 pm - 3:15 pm &  Receso   \\
3:15 pm - 4:15 pm    &  Discusión de problemas   \\
4:15 pm -4:45 pm & Examen corto \\
4:45 pm - 5:00 pm &  Receso   \\
5:00 pm - 6:00 pm & Tareas y resolución de problemas.\\
\hline
\end{tabular}
\end{table}

\end{center}
%\begin{center}
  
% \begin{tabular}{|c|c|}  	\hline
%    2:00 pm - 3:45 pm & Clase Te\'orica   \\  \arrayrulecolor{yellowgreen}  \hline
%    3:45 pm - 4:15 pm  & Receso        \\  \arrayrulecolor{yellowgreen}   \hline
%   4:15 pm - 5:00 pm  & Resoluci\'on de problemas.   \\  \arrayrulecolor{yellowgreen}    \hline
% 5:00 pm - 6:00 pm  & Evaluaci\'on.  \\  \arrayrulecolor{yellowgreen}    \hline
%\end{tabular}

%\end{center}


\thispagestyle{empty}
\item {\bf\large VIII.  BIBLIOGRAFÍA.}
\begin{itemize}
\item Flores, Jorge; Meléndez Rodrigo. Cuaderno de Apuntes nivel 5 2019.
\item Kenneth H. Rosen. Discrete Mathematics and Its applications. Mc Graw Hill.
\item José Heber Nieto Said. Combinatoria para Olimpiadas Matemáticas.
\item José Heber Nieto Said. Teoría Combinatoria.
\item Soberón, Pablo. Combinatorias para olimpiadas. 2010.
\item Chen Chuan-Chong and Koh Khee-MengPinciples and Tecniques in Combinatorics.
\item Pérez Seguí, María Luisa Combinatorias para olimpiadas. Combinatoria cuadernos de olimpiadas de matemática
\item L. Lováz; J. Pelikan; K. Vesztergombi Discrete Mathematics: elementary and Beyond.
Springer.
\item Ralph P. Grimaldi. Matemáticas Discretas y Combinatoria. Una introducción con aplicaciones.

\end{itemize}

\end{enumerate}




