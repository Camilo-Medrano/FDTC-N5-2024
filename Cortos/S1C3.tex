\begin{problema}
En un plano hay $n$ puntos, $k$ de los cuales están alineados. A excepción de ellos no hay tres en línea recta. ¿Cuántas líneas rectas diferentes resultan si se unen los $n$ puntos dos a dos?
\end{problema}

\begin{problema}
En un tablero de $5\times 5$ tratamos de contar todos los posibles rectángulos cuyos lados son paralelos a los bordes del tablero. Dos rectángulos se considerarán diferentes si son diferentes susdimensiones o las posiciones que ocupan son diferentes. ¿Cuántos rectángulos son? ¿Cuántos rectángulos serían si el tablero fuera de $n\times n$?
\end{problema}

\textbf{Crédito extra} Pruebe que el número de subconjuntos de $\mathbb{N}_n$ con $k$ elementos y sin enteros consecutivos es igual $\displaystyle \binom{n-k+1}{k}$.