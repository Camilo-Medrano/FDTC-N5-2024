\documentclass[12pt]{article}

%Paquetes a utilizarse
\usepackage[width=7in, height=9.5in, top=0.75in, papersize={8.5in,11in}]{geometry}
\usepackage[spanish]{babel} 
\decimalpoint
\usepackage[utf8]{inputenc}
\usepackage{bbding}
\usepackage[colorlinks = true, linkcolor = blue, urlcolor = BlueViolet, citecolor = OliveGreen]{hyperref}
\usepackage{graphicx}
\usepackage{amssymb,amsthm,amsmath}
\usepackage{enumerate}
\usepackage{array,multicol,multirow}
\usepackage{xcolor}
\usepackage{fancybox,tcolorbox}
\usepackage{caption,subcaption,float,tabularx}
\usepackage{enumitem}

\theoremstyle{definition}
\newtheorem{corolario}{Corolario}
\newtheorem{lema}[corolario]{Lema}
\newtheorem{proposicion}[corolario]{Proposición}
\newtheorem{teorema}[corolario]{Teorema}
\newtheorem{propiedad}[corolario]{Propiedad}
\newtheorem*{observacion}{Observación}
\newtheorem{definicion}{Definición}
\newtheorem*{demostracion}{Demostración}
\newtheorem{ejemplo}{Ejemplo}
\newtheorem{problema}{Problema}
\newtheorem*{solucion}{Solución}
\newtheorem{ejercicio}{\PencilRightDown \  Ejercicio}
\newtheorem{step}{Paso}
\newtheorem{credito}{Crédito}

\usepackage{tikz}
\usetikzlibrary{arrows.meta,babel,calc,positioning}

\renewcommand{\arraystretch}{1.5}
\providecommand{\abs}[1]{\lvert#1\rvert}
\providecommand{\norm}[1]{\lVert#1\rVert}

\renewcommand{\tabularxcolumn}[1]{m{#1}}
\newcommand{\Evaluacion}[4]{
\setcounter{ejercicio}{0}
\noindent\begin{tabular}{lcr}
	\includegraphics[height=3cm]{Logos/logo-UES.png}\hspace{2.5em}
	&
	\includegraphics[height=2.75cm]{Logos/logo-PJT.png}
	& 
	\hspace{2.5em}\includegraphics[height=2.75cm]{Logos/logo-MINEDUCYT.png}
\end{tabular}

\hfill

\begin{center}
    
    UNIVERSIDAD DE EL SALVADOR
    \\PROGRAMA JÓVENES TALENTO
    \\FDTC 2022
    \\#2
    \\Nivel Olímpico C de Matemáticas

\end{center}

\begin{center}
    #1
\end{center}

%\textbf{Nombre}: \enspace\hrulefill

#3

\input{#4}
\newpage
}

\newtheorem{obs}{Observación}

%\usepackage[margin=2.5cm]{geometry}
%\usepackage{wasysym}
%\usepackage{stmaryrd,textcomp}
%\usepackage{pgf,tikz}
%\usetikzlibrary{arrows}

\parskip = 2mm   %%%% genera un espacio de X mm entre lo párrafos
\parindent = 3mm
\usepackage{multicol}
\usepackage{iwona}

\newcommand{\tema}{Introducción}
\newcommand{\fecha}{Martes, 29 de noviembre de 2022}
\newcommand{\sesion}{Tarea 1}

\begin{document}
%\thispagestyle{empty}
%\newpage
\thispagestyle{empty}

\begin{figure}[h] 
	\begin{minipage}[b]{0.26\textwidth}
		\begin{center}
			\includegraphics[height=3cm]{Logos/UES.png}
			\par\end{center}
	\end{minipage} 
	\begin{minipage}[b]{0.46\textwidth}
		\begin{center}
			UNIVERSIDAD DE EL SALVADOR\\ [0.1cm]
			PROGRAMA JÓVENES TALENTO\\ [0.1cm]
	        FDTC 2022\\ [0.1cm]
                NIVEL 5\\ [0.1cm]
			COMBINATORIA 
			\par\end{center}
	\end{minipage} 
	\begin{minipage}[b]{0.05\textwidth}
		\begin{center}
			\includegraphics[height=2cm]{Logos/LOGO PJT.png}
			\par\end{center}
	\end{minipage}
\end{figure}

\begin{center}
    \begin{tabular}{p{4.5cm} p{7cm} p{4.5cm}}
        \tema & \centering\fecha & \hfill\sesion
    \end{tabular}
\end{center}

\begin{problema}
    Manuel tiene cuatro carritos de juguete de diferentes colores (azul, blanco, verde y rojo, uno de cada color) y decide regalárselos a sus hermanos Luis y Karen. ¿De cuántas maneras diferentes puede hacer esto? Ejemplo: podría dar los cuatro carritos a Luis. También podría dar tres de ellos a Karen y el otro a Luis.
\end{problema}

\begin{problema}
    Se tienen los conjuntos $P=\{ x \ | \ x \ es \ una \ letra \ de \ la \ palabra \ ''guadalajara''\}$ y $Q=\{ x \ | \ x \ es \ una \ letra \ de \ la \ palabra \ ''carcajada''\}$, determine:

    \renewcommand{\labelenumi}{\alph{enumi})}
    \begin{multicols}{2}
    \begin{enumerate}
        \item $|P|$
        \item $P \cap Q$
        \item $|P \cup Q|$
        \item $P-Q$
        \item $Q-P$
    \end{enumerate}
    \end{multicols}
\end{problema}

\textbf{Crédito extra:} a un grupo de estudiantes, se le preguntó por sus colores favoritos. La información recabada fue la siguiente: a 33 les gusta el azul, a 32 les gusta el rojo, a 28 les gusta el amarillo, a 11 les gusta el azul y el rojo, a 15 les gusta el azul y el amarillo, a 14 les gusta el rojo y el amarillo, a 5 les gustan los tres, y a 7 no les gusta ninguno. Determine:

    \renewcommand{\labelenumi}{\alph{enumi})}
    \begin{enumerate}
        \item ¿A cuántos estudiantes les gusta únicamente el amarillo?
        \item ¿A cuántos estudiantes les gustan exactamente dos colores?
        \item ¿Cuál es el número total de estudiantes?
    \end{enumerate}

\textbf{Recomendación:} apoyarse de un diagrama de Venn.

\end{document}