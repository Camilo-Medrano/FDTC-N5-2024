\section{Combinaciones}

Hemos dicho que un conjunto \textit{es una colección no ordenada de elementos distintos}, ahora dado un conjunto de $n$ elementos supongamos que necesitamos seleccionar $k\leq n$ elementos no ordenados de este, ¿de cuantas maneras podemos hacer esto? más concretamente ¿cuántos subconjuntos hay de $k$ elementos en un conjunto dado de $n$ elementos con $k\leq n$? este número se le conoce como número combinatorio y se denota por como $\displaystyle C_k^n$, \( C(n, k) \)  o como es frecuente $\displaystyle {n \choose k}$ acontinuación presentamos diversas modelos que nos llevan a dicha definición.

\subsection{Modelo conjuntista}

\begin{definicion}
 En el modelo conjuntista, las combinaciones se representan mediante subconjuntos. Supongamos que tenemos un conjunto \( A \) con \( n \) elementos. La combinación de \( k \) elementos de \( A \) y se refiere a todos los subconjuntos de \( A \) con exactamente \( k \) elementos.   
\end{definicion}

\begin{ejemplo}
Si \( A = \{a, b, c\} \), entonces \( C(3, 2) \) representa todas las combinaciones posibles de elegir 2 elementos de \( A \). Estas combinaciones serían \(\{a, b\}\), \(\{a, c\}\), y \(\{b, c\}\).
\end{ejemplo}

\begin{ejemplo}
Hallar $\displaystyle\binom{n}{1}, \displaystyle\binom{n}{n-1}, \binom{n}{n}$ y $\displaystyle\binom{n}{0}$. 
\end{ejemplo}

\begin{solucion}
Para los subconjuntos de solo un elemento tenemos $n$ opciones claramente, así $\displaystyle\binom{n}{1}=n$. Para cada subconjunto de $n-1$ elemento existe uno y solo un subconjunto unitario cuyo elemento es el que no se escogió en los $n-1$, así contar todos los subconjuntos de $n-1$ elementos equivale a contar los subconjunto de tamaño $1$, luego $\displaystyle\binom{n}{n-1}=n$. El único subconjunot de $n$ elementos en un conjunto de tamaño $n$ es el mismo, así como también el único subconjunto que contiene cero elementos es el subconjunto vacío, luego $\displaystyle\binom{n}{n}=\displaystyle\binom{n}{0}=1$.     
\end{solucion}

\subsection{Fórmula de Combinaciones}

La fórmula para calcular el número de combinaciones \( C(n, k) \) es:

\[ C_{k}^n=C(n, k) = \binom{n}{k} = \frac{n!}{k!(n-k)!} \]

La fórmula refleja la cantidad de maneras diferentes en las que puedes elegir \( k \) elementos de un conjunto de \( n \) elementos, sin tener en cuenta el orden.

\begin{ejemplo}
Supongamos que tienes un conjunto \( B = \{a, b, c, d\} \), y deseas encontrar \( C(4, 2) \), es decir, el número de combinaciones de 2 elementos de \( B \). Aplicamos la fórmula:

\[ \binom{4}{2} = \frac{4!}{2!(4-2)!} = \frac{24}{2 \cdot 2} = 6 \]
Por lo tanto, hay $6$ maneras diferentes de elegir 2 elementos de \( B \).
\end{ejemplo}\vspace{0.3cm}

\begin{ejemplo}
Supongamos que tienes un conjunto de 5 puntos no colineales en un plano, etiquetados como \( P_1, P_2, P_3, P_4, P_5 \). Queremos encontrar el número de formas en que puedes elegir 3 puntos para formar un triángulo. La solución es \( C(5, 3) \). Veamos,

\[ \binom{5}{3} = \frac{5!}{3!(5-3)!} = \frac{120}{6 \cdot 2} = 10 \]
Entonces, hay 10 formas diferentes de elegir 3 puntos para formar un triángulo con estos 5 puntos (intenta esbozar algunos triángulos).
\end{ejemplo}\vspace{0.3cm}

\begin{ejemplo}
Imagina que tienes 4 líneas en un plano no paralelas dos a dos, numeradas como \( L_1, L_2, L_3, L_4 \). Queremos saber de cuántas maneras puedes elegir 2 líneas para formar un cruce. 
\end{ejemplo}

\begin{solucion}
La respuesta es \( C(4, 2) \).

\[ \binom{4}{2} = \frac{4!}{2!(4-2)!} = \frac{24}{2 \cdot 2} = 6 \]

Así que hay 6 maneras diferentes de seleccionar 2 líneas para formar un cruce con estas 4 líneas.
\end{solucion}



\begin{ejemplo}
Supongamos que tienes 6 segmentos de línea en un plano, representados como \[ S_1, S_2, S_3, S_4, S_5, S_6\;. \] Queremos determinar cuántas formas existen de seleccionar 4 segmentos para formar un cuadrilátero. 
\end{ejemplo}

\begin{solucion}
La solución es \( C(6, 4) \).

\[ \binom{6}{4} = \frac{6!}{4!(6-4)!} = \frac{720}{24} = 30 \]
Por lo tanto, hay 30 maneras diferentes de elegir 4 segmentos para formar un cuadrilátero con estos 6 segmentos. 
\end{solucion}



\begin{ejemplo}
Supongamos que cuentas con un grupo de 8 voluntarios, etiquetados como \(V_1, V_2, \ldots, V_8\). Este será nuestro conjunto base.
\end{ejemplo}

\begin{solucion}
\[ A = \{V_1, V_2, V_3, V_4, V_5, V_6, V_7, V_8\} \]
Decides formar equipos de 3 voluntarios cada uno. Entonces, en términos combinatorios, estás buscando todas las combinaciones posibles de grupos de 3 voluntarios seleccionados de tu conjunto \(A\).\\\\
\[ \binom{8}{3} = \frac{8!}{3!(8-3)!} = \frac{8!}{3! \cdot 5!} =56\]
Por lo tanto, hay 56 formas únicas de formar equipos de 3 voluntarios cada uno a partir del grupo de 8 voluntarios disponibles.   
\end{solucion}

\begin{ejemplo}
Calcular $\displaystyle \binom{n}{2}$.
\end{ejemplo}

\begin{solucion}
Fácilmente se puede calcular usando la fórmula para combinaciones, no obstante daremos otro método de solución dado que es conveniente en los problemas desarrollar algunas estrategias para contar.

Supongamos que el conjunto $A=\{a_1, a_2, a_3, \cdots a_n$. Usaremos recurrencia y el principio de la suma para esta prueba; Sea $S_k$ el total de subconjuntos de dos elementos que hay en un subconjunto de $k$ elementos. Los suconjuntos de dos elementos se pueden clasificar en los subconjuntos que contienen al n-ésimo elemento $a_n$ y los subconjuntos de dos elementos que no lo contienen, notemos que hay $n-1$ subconjuntos que contienen a $a_n$, pues para cada subconjunto de dos elementos solo debemos seleccinar un elemento del conjunto dado que $a_n$ está fijo. Además para los subconjuntos de dos elementos que no contienen a $a_n$ equivale hallar los subconjuntos de cardinalidad 2 de un subconjunto de $n-1$ elementos a saber $\{a_1, a_2, a_3, \cdots a_{n-1}$. Entonces tenemos la fórmula de recurrencia $$\displaystyle \binom{n}{2}=S_n = (n-1)+S_{n-1}.$$

Dado que $S_2=\displaystyle \binom{2}{2}=1$, tenemos 
\begin{eqnarray*}
\displaystyle \binom{n}{2} & = & S_n = (n-1)+S_{n-1}\\
& = & (n-1)+(n-2)+S_{n-2}\\
&\vdots & \\
& = & (n-1)+(n-2)+(n-3)+\cdots 2+1=\dfrac{n(n-1)}{2} 
\end{eqnarray*}
\end{solucion}

\begin{teorema}
    Para todo $n$ se cumple la siguiente identidad $$2^n=\binom{n}{0}+\binom{n}{1}+\binom{n}{2}+\cdots +\binom{n}{n-1}+\binom{n}{n}.$$
\end{teorema}
\begin{demostracion}
    Se deja como ejercicio al lector.
\end{demostracion}

\subsection{Resultados sobre combinaciones}
\begin{teorema}[Teorema de Pascal]
Para cualquier número entero positivo \(n\) y \(k\) tal que \(0 \leq k \leq n\), se cumple que \[\displaystyle \binom{n}{k} = \binom{n-1}{k-1}+ \binom{n-1}{k}\;.\]
\end{teorema}

\begin{demostracion}
Seleccionemos un elemento particular $p$ del conjunto, ahora los dos posibles casos disjuntos para seleccionar subconjuntos de tamaño $k$ son: que se incluya el elemento particular $p$ seleccionado o que no se incluya, entonces por el principio de la suma tenemos $$\binom{n}{k} = \{\text{subconjuntos de tamaño}\, k\, \text{que contienen a}\,p\}+\{\text{subconjuntos de tamaño}\, k\, \text{que no contienen a}\,p\}.$$

Pero el total de subconjuntos de tamaño $k$ que no contienen a $p$ es $\displaystyle\binom{n-1}{k}$, mientras que si $p$ debe ser un elemento fijo en cada subconjunto de tamaño $k$ solo debemos seleccionar $k-1$ elementos de los $n-1$ posibles dado que no podemos volver a seleccionar $p$. Por lo tanto el total de subconjuntos que contienen a $p$ es $\displaystyle\binom{n-1}{k-1}$ y por consiguiente el teorema está probado.
\end{demostracion}

\begin{demostracion}
(Utilizando la fórmula de combinaciones \(\displaystyle \binom{n}{k} = \frac{n!}{k!(n-k)!}\)).

\[\binom{n-1}{k} = \frac{(n-1)!}{k!(n-k-1)!} \quad \text{y} \quad \binom{n-1}{k-1} = \frac{(n-1)!}{(k-1)!(n-k)!}\]
Ahora:
\begin{eqnarray*}
    \binom{n-1}{k-1}+\binom{n-1}{k}&=&\frac{(n-1)!}{(k-1)!(n-k)!}+\frac{(n-1)!}{k!(n-k-1)!}\\
    &=&\frac{k\cdot (n-1)!}{k\cdot(k-1)!(n-k)!}+\frac{(n-k)\cdot(n-1)!}{(n-k)\cdot k!(n-k-1)!}\\
    &=& \frac{k\cdot (n-1)!}{k!(n-k)!}+\frac{(n-k)\cdot (n-1)!}{k!(n-k)!}\\
    &=&\frac{k(n-1)!+(n-k)(n-1)!}{k!(n-k)!}\\
    &=&\frac{\cancel{k(n-1)!}+n(n-1)!-\cancel{k(n-1)!}}{k!(n-k)!}\\
    &=&\frac{n!}{k!(n-k)!}\\
    &=&\binom{n}{k}
\end{eqnarray*}
\end{demostracion}

\begin{teorema}[Ley de simetría]
Para cualquier número entero no negativo \( n \) y \( k \) tal que \( 0 \leq k \leq n \), se cumple que \[ \binom{n}{k} = \binom{n}{n-k} \].
\end{teorema}

\begin{demostracion}
(Utilizando el principio de biyección)\\
Para cada subconjunto de $k$ elementos hay exactamente un subconjunto de $n-k$ elementos que no contiene los $k$ elementos seleccionados, así contar el total de subconjuntos de $k$ elementos de un conjunto de $n$ elementos es equivalente a contar todos los subconjuntos de $n-k$ elementos que hay en este.
\end{demostracion}

\begin{demostracion}
(Utilizando la fórmula).
\[ \binom{n}{n-k} = \frac{n!}{(n-k)![n-(n-k)]!} = \frac{n!}{k!(n-k)!} = \binom{n}{k} \]
\end{demostracion}

\begin{teorema}[Teorema de Vandermonde]
    Para cualesquiera números enteros no negativos \( m, n, \) y \( r \), se cumple que

\[ \binom{n+m}{r} = \sum_{k=0}^{r} \binom{n}{k} \cdot \binom{m}{r-k} \]
\end{teorema}

\begin{demostracion}[Idea de la prueba]

Sea $A$ y $B$ conjuntos disjuntos tal que $|A|=n$ y $|B|=m$, entonces $|A \cup B|=n+m$.; Supongamos ahora que queremos formar un subconjunto de $r$ elementos de modo que $k\leq r$ sean elementos del conjunto $A$ y $r-k$ sean elementos del conjunto $B$, ¿cuántos subconjuntos hay de este tipo? podemos escoger $k$ elmentos del conjunto $A$ de $\displaystyle\binom{n}{k}$ formas mientras que podemos escoger $r-k$ elementos del conjunto $B$ de $\displaystyle\binom{m}{r-k}$ formas, así hay por el principio de la multiplicación $$\binom{n}{k}\binom{m}{r-k}$$ subconjuntos de $r$ elementos donde $k$ son elementos de $A$ y $r-k$ elementos de $B$.  Para hallar todos los subconjuntos de $r$ elementos de $A\cup B$ podemos separarlo en casos disjuntos para cada $k$, entonces el total de subconjuntos de $r$ elementos de $A\cup B$ son el total de subconjuntos con $0$ elementos en $A$ y $k$ elementos en $B$, más el total de subconjuntos con $1$ elemento en $A$ y $k-1$ elementos en $B$, más el total de subconjuntos con $2$ elemento en $A$ y $k-2$ elementos en $B$, así sucesivamente hasta llegar a solo contar los subconjuntos con $k$ elementos en $A$ y ningún elemento en $B$, cada uno de estos casos son disjuntos y por lo tanto tenemos 
\[ \binom{n+m}{r} = \sum_{k=0}^{r} \binom{n}{k} \cdot \binom{m}{r-k}. \]
%Supongamos que tenemos dos grupos de estudiantes: \(m\) estudiantes en el Grupo A y \(n\) estudiantes en el Grupo B. Queremos formar un comité de \(r\) estudiantes seleccionados de ambos grupos. Demostraremos que el número total de formas de formar este comité es igual a \(\displaystyle \binom{m+n}{r}\). La identidad de Vandermonde nos dice que el número de formas de seleccionar \(r\) estudiantes de \(m+n\) candidatos es igual a:

%\[ \sum_{k=0}^{r} \binom{m}{k} \binom{n}{r-k} \]
%Esto representa la suma de todas las formas posibles de seleccionar \(k\) estudiantes del Grupo A y \(r-k\) estudiantes del Grupo B para formar el comité.\\\\
%El número total de formas de seleccionar \(r\) estudiantes de \(m+n\) candidatos es simplemente \(\displaystyle \binom{m+n}{r}\).\\\\
%Ambos enfoques deben dar el mismo resultado. 

Realizamos una prueba numérica para \(m = 3\), \(n = 2\) y \(r = 2\):

\[ \sum_{k=0}^{2} \binom{3}{k} \binom{2}{2-k} \stackrel{?}{=} \binom{5}{2} \]
Calculamos ambos lados:

\[ \text{Lado izquierdo:} \quad \binom{3}{0} \binom{2}{2} + \binom{3}{1} \binom{2}{1} + \binom{3}{2} \binom{2}{0} \]

\[ = 1 \cdot 1 + 3 \cdot 2 + 3 \cdot 1 = 1 + 6 + 3 = 10 \]

\[ \text{Lado derecho:} \quad \binom{5}{2} = \frac{5!}{2!3!} = \frac{120}{12} = 10 \]
Ambos lados dan el mismo resultado, confirmando la identidad de Vandermonde en este caso específico.
\end{demostracion}

\subsection{Modelo de cadenas binarias}
Hasta el momento se ha dicho que $\displaystyle\binom{n}{k}$ puede interpretarse como la cantidad de subconjuntos de $k$ elementos de un conjunto de cardinalidad $n$, es decir, la cantidad de formas de escoger $k$ objetos entre $n$ objetos diferentes. Ahora bien, esta interpretación de un número combinatorio puede asociarse a otro problema:¿Cuántas cadenas de unos y ceros se pueden formar con $k$ unos y $n-k$ ceros?

Podemos interpretar esto como tener a disposición $n$ espacios vacíos en los que se debe colocar o bien un 1 o bien un 0.

\begin{center}
    $\underbrace{1\ 0\ 0\ 1\ 0\ 1...0\ 1\ 1\ 0}_{n}$
\end{center}

Además se tiene prefijada la cantidad de unos y ceros que deben colocarse en esos $n$ espacios, en total son exactemente $k$ unos y, por ende, los restantes $n-k$ deben ser ceros.  Entonces basta conocer de cuántas formas puedo colocar los unos y luego las posiciones de los ceros quedan determinadas. Es así como se tiene que
la cantidad de cadenas de longitud $n$ de $k$ unos y $n-k$ ceros está dada por $\displaystyle\binom{n}{k}$.

\begin{ejemplo}
Pruebe usando cadenas de ceros o unos que:
\[\binom{n}{k}=\binom{n-2}{k-2}+2\binom{n-2}{k-1}+\binom{n-2}{k}\]
\end{ejemplo}

\begin{solucion}
    Asumamos que $k$ es la cantidad de unos, así que $\binom{n}{k}$ cuenta de \textit{golpe} la cantidad de tales cadenas de longitud $n$. Bajo la misma idea, ¿qué podemos decir de los 3 términos de la derecha?

    Para $\displaystyle \binom{n-2}{k-2}$ son las cadenas de longitud $n-2$ pero con $k-2$ unos. Para hacerlas de longitud $n$ con $k$ unos. se agrega al final de estas dos veces $1$, al hacerlo obtenemos el total de cadenas de longitud $n$ conjunos pero que terminan en $\ldots 11$.

    Para $\displaystyle \binom{n-2}{k-1}$ son las cadenas de longitud $n-1$ pero con $k-1$ unos. Para hacerla de longitud $n$ con $k$ unos, se agrega al final de estas $01$ o $10$. De manera que hay dos formas para las cadenas calculadas, por ello el coeficiente $2$.

    Para $\binom{n-2}{k}$ son las cadenas de longitud $n-2$ con $k$ unos. Para hacerla de longitud $n$, debemos llenar dos espacios con ceros, es decir, 00, ya que los unos están completos.

    Como cada caso es excluyente del resto, por el principio de la suma tenemos el resultado:
\[\binom{n}{k}=\binom{n-2}{k-2}+2\binom{n-2}{k-1}+\binom{n-2}{k}\]
\end{solucion}

\section{Problemas propuestos}
\begin{problema}
En el alfabeto Morse, usado en telegrafía, se emplean solamente dos signos: el punto y la raya. ¿Cuántas palabras distintas pueden formarse compuestas de uno, dos, tres, cuatro o cinco signos? Generalice.
\end{problema}

\begin{problema}
¿Cuántos números mayores que $3000$ y menores que $4000$ pueden formarse con los dígitos $2$, $3$, $5$ y $7$ a) si cada cifra puede usarse sólo una vez;
b) si cada cifra puede emplearse las veces que se desee.
\end{problema}

\begin{problema}
Si se forman todos los números que resultan de permutar las cifras de $123579$ y se ordenan en forma creciente, ¿qué lugar ocupará el número $537192$?
\end{problema}

\begin{problema}
Encuentra el número de cuádruplas ordenadas $(x_1,x_2,x_3,x_4)$ de números impares positivos que satisfacen la ecuación $x_1+x_2+x_3+x_4=98$.
\end{problema}

\begin{problema}
    ¿En cuántas formas se pueden ordenar los números $21$, $31$, $41$, $51$, $61$, $71$ y $81$ de tal manera que la suma de cuatro enteros consecutivos son divisibles por $3$?
\end{problema}

\begin{problema}
Considere todos los números posibles de 8 cifras diferentes no nulas (como, por ejemplo, $73451962$).
\begin{enumerate}
    \item ¿Cuántos de ellos son divisibles entre $5$?
    \item ¿Cuántos de ellos sin divisibles entre $9$?
\end{enumerate}
\end{problema}

\begin{problema}
¿Cuántos triángulos se pueden formar que tengan como vértices los vértices de un decágono regular?
\end{problema}

\begin{problema}
Un número es \textit{capicúa} si se lee igual de izquierda a derecha que de derecha a izquierda. Por ejemplo $7$, $33$ y $252$ son capicúas. ¿Cuántos capicúas hay desde $1$ hasta $2023$?
\end{problema}

\begin{problema}
Probar la identidad combinatoria dada en el teorema 1, usando:
\begin{enumerate}
    \item La fórmula para combinaciones.
    \item El principio de la suma.
\end{enumerate}
\end{problema}

\begin{problema}
    Nueve sillas en fila deben ser ocupadas por seis estudiantes y los profesores Alpha, Beta y Gamma. Estos tres profesores llegan antes que los seis estudiantes y deciden elegir sus sillas de manera que cada profesor esté entre dos estudiantes. ¿De cuántas maneras pueden los profesores Alpha, Beta y Gamma elegir sus sillas?
\end{problema}

\begin{problema}
Si $n$ puntos distintos situados en una circunferencia se unen de todas las maneras posibles, ¿cuántos puntos de intersección resultan, como máximo?
\end{problema}

\begin{problema}
Para escribir todos los números naturales desde $1$ hasta $1000000$, ¿cuántos ceros se necesitan?
\end{problema}

\begin{problema}
    Probar que 
    \[k\binom{n}{k}=n\binom{n-1}{k-1}\]
\end{problema}

\begin{problema}
Pruebe que todo $n$ se cumple la siguiente identidad $$\binom{n}{0}-\binom{n}{1}+\binom{n}{2}-\cdots +(-1)^{n-1}\binom{n}{n-1}+(-1)^n\binom{n}{n}=0.$$    
\end{problema}

\begin{problema}
En un acto deben hablar $n$ mujeres y $k$ hombres. ¿De cuántas maneras se puede ordenar la lista de oradores con la condición de que no hablen dos hombres consecutivamente?
\end{problema}

\begin{problema}
    Dos de los cuadrados de un área de tablero de ajedrez de $7\times 7$ están pintados de amarillo, y el resto están pintados de verde. Dos esquemas de colores son equivalentes si uno se puede obtener a partir del otro aplicando una rotación en el plano del tablero. ¿Cuántos esquemas de colores no equivalentes son posibles?
\end{problema}


\begin{problema}
Marcos juega un juego de computador en una cuadrícula de $4\times 4$. Cada celda es roja o azul, pero el color sólo se ve si se hace clic en ella. Se sabe que sólo hay dos celdas azules, y que tienen un lado común.
¿Cuál es el menor número de clics que Marcos tiene que hacer para estar seguro de ver las dos celdas azules en la pantalla?
\end{problema}

\end{document}