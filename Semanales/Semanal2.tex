\vspace{0.2cm}
\begin{enumerate}
    \item (25\%) Demuestre que, dados $5$ puntos en el plano con coordenadas enteras, siempre hay dos tales que su punto medio tiene coordenadas enteras. Tenga en cuenta que las coordenadas del punto medio de dos puntos, $A=(x_1,y_1)$ y $B=(x_2,y_2)$, están dadas por $\displaystyle M=\left(\frac{x_1 + x_2}{2}, \frac{y_1 + y_2}{2}\right)$.
    \item (25\%) Para los enteros positivos  $1,2,3,\dots ,n$ hay 11660 desarreglos donde $1,2,3,4,5$ aparecen en los primeros cinco lugares. ¿Cuál es el valor de $n$?
    \item (25\%) Imagina que tienes cuatro monedas diferentes: una de 1 euro, una de 50 centavos, una de 20 centavos y una de 10 centavos. ¿De cuántas maneras diferentes puedes colocar estas monedas en una fila si la moneda de 50 centavos siempre debe estar separada de la moneda de 1 euro por al menos una moneda?
    \item (25\%) Utilizando el método de contar de dos formas, demuestra la siguiente propiedad del Hockey Stick (o Christmas Stocking Theorem) en el Triángulo de Pascal: 

\textit{La suma de los números combinatorios a lo largo de una diagonal del Triángulo de Pascal, comenzando con \( \displaystyle\binom{r}{r} \) y extendiéndose hacia abajo hasta \( \displaystyle\binom{n}{r} \), es igual a \( \displaystyle \binom{n+1}{r+1} \).}

Matemáticamente, se expresa como:
\[ \binom{r}{r} + \binom{r+1}{r} + \binom{r+2}{r} + \cdots + \binom{n}{r} = \binom{n+1}{r+1} \]
\end{enumerate}

\textbf{Crédito Extra:} Dada la expresión \[\left(7x^2+\frac{1}{2x}\right)^{25}\]encuentre la cantidad de términos de su expansión, la suma de los coeficientes de su expansión, el coeficiente de $x^{13}$ si es posible, y encuentre el i-ésimo término