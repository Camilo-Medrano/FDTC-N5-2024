\documentclass[12pt]{article}

%Paquetes a utilizarse
\usepackage[width=7in, height=9.5in, top=0.75in, papersize={8.5in,11in}]{geometry}
\usepackage[spanish]{babel} 
\decimalpoint
\usepackage[utf8]{inputenc}
\usepackage{bbding}
\usepackage[colorlinks = true, linkcolor = blue, urlcolor = BlueViolet, citecolor = OliveGreen]{hyperref}
\usepackage{graphicx}
\usepackage{amssymb,amsthm,amsmath}
\usepackage{enumerate}
\usepackage{array,multicol,multirow}
\usepackage{xcolor}
\usepackage{fancybox,tcolorbox}
\usepackage{caption,subcaption,float,tabularx}
\usepackage{enumitem}

\theoremstyle{definition}
\newtheorem{corolario}{Corolario}
\newtheorem{lema}[corolario]{Lema}
\newtheorem{proposicion}[corolario]{Proposición}
\newtheorem{teorema}[corolario]{Teorema}
\newtheorem{propiedad}[corolario]{Propiedad}
\newtheorem*{observacion}{Observación}
\newtheorem{definicion}{Definición}
\newtheorem*{demostracion}{Demostración}
\newtheorem{ejemplo}{Ejemplo}
\newtheorem{problema}{Problema}
\newtheorem*{solucion}{Solución}
\newtheorem{ejercicio}{\PencilRightDown \  Ejercicio}
\newtheorem{step}{Paso}
\newtheorem{credito}{Crédito}

\usepackage{tikz}
\usetikzlibrary{arrows.meta,babel,calc,positioning}

\renewcommand{\arraystretch}{1.5}
\providecommand{\abs}[1]{\lvert#1\rvert}
\providecommand{\norm}[1]{\lVert#1\rVert}

\renewcommand{\tabularxcolumn}[1]{m{#1}}
\newcommand{\Evaluacion}[4]{
\setcounter{ejercicio}{0}
\noindent\begin{tabular}{lcr}
	\includegraphics[height=3cm]{Logos/logo-UES.png}\hspace{2.5em}
	&
	\includegraphics[height=2.75cm]{Logos/logo-PJT.png}
	& 
	\hspace{2.5em}\includegraphics[height=2.75cm]{Logos/logo-MINEDUCYT.png}
\end{tabular}

\hfill

\begin{center}
    
    UNIVERSIDAD DE EL SALVADOR
    \\PROGRAMA JÓVENES TALENTO
    \\FDTC 2022
    \\#2
    \\Nivel Olímpico C de Matemáticas

\end{center}

\begin{center}
    #1
\end{center}

%\textbf{Nombre}: \enspace\hrulefill

#3

\input{#4}
\newpage
}

\newtheorem{obs}{Observación}

%\usepackage[margin=2.5cm]{geometry}
%\usepackage{wasysym}
%\usepackage{stmaryrd,textcomp}
%\usepackage{pgf,tikz}
%\usetikzlibrary{arrows}

\parskip = 2mm   %%%% genera un espacio de X mm entre lo párrafos
\parindent = 3mm
\usepackage{multicol}
\usepackage{iwona}

\newcommand{\tema}{Semana 2}
\newcommand{\fecha}{Sábado, 10 de diciembre de 2022}
\newcommand{\sesion}{Examen semanal}

\begin{document}
%\thispagestyle{empty}
%\newpage
\thispagestyle{empty}

\begin{figure}[h] 
	\begin{minipage}[b]{0.26\textwidth}
		\begin{center}
			\includegraphics[height=3cm]{Logos/UES.png}
			\par\end{center}
	\end{minipage} 
	\begin{minipage}[b]{0.46\textwidth}
		\begin{center}
			UNIVERSIDAD DE EL SALVADOR\\ [0.1cm]
			PROGRAMA JÓVENES TALENTO\\ [0.1cm]
	        FDTC 2022\\ [0.1cm]
                NIVEL 5\\ [0.1cm]
			COMBINATORIA 
			\par\end{center}
	\end{minipage} 
	\begin{minipage}[b]{0.05\textwidth}
		\begin{center}
			\includegraphics[height=2cm]{Logos/LOGO PJT.png}
			\par\end{center}
	\end{minipage}
\end{figure}

\begin{center}
    \begin{tabular}{p{4.5cm} p{7cm} p{4.5cm}}
        \tema & \centering\fecha & \hfill\sesion
    \end{tabular}
\end{center}



{\bf PARTE I (20 \%):}\\
\textit{Determine si las siguientes afirmaciones son falsas o verdaderas e indique con una $V$ en el caso de ser verdadera o una $F$ en el caso de ser falsa. Las afirmaciones no necesitan argumentación.}

  \begin{enumerate}

  \item (5\%) $\left( n-\cfrac{1}{n}\right)^2=n^2+1/n^2-2$\dotfill{\bf \rule{1.5cm}{0.7pt}}
\item (5\%) $2^{2022}=2+C^1_{2022}+C^2_{2022}+C^3_{2022}+\cdots +C^{2021}_{2022}$ \dotfill{\bf \rule{1.5cm}{0.7pt}}
\item (5\%) En la figura, el número de caminos para llegar del punto $A$  al punto $B$ es $C^2_2+C^2_6+C^2_3$ \dotfill{\bf \rule{1.5cm}{0.7pt}}

\begin{figure}[H]
    \centering
    \scalebox{0.35}{\begin{tikzpicture}[line cap=round,line join=round,>=triangle 45,x=1cm,y=1cm]
\clip(-9.74,-0.36) rectangle (0.8,9.2);
\draw [line width=1.2pt] (-7,3)-- (-7,1);
\draw [line width=1.2pt] (-7,1)-- (-5,1);
\draw [line width=1.2pt] (-5,1)-- (-5,3);
\draw [line width=1.2pt] (-7,3)-- (-5,3);
\draw [line width=1.2pt] (-5,7)-- (-5,3);
\draw [line width=1.2pt] (-5,3)-- (-3,3);
\draw [line width=1.2pt] (-5,7)-- (-3,7);
\draw [line width=1.2pt] (-3,7)-- (-3,3);
\draw [line width=1.2pt] (-3,8)-- (-3,7);
\draw [line width=1.2pt] (-3,7)-- (-1,7);
\draw [line width=1.2pt] (-3,8)-- (-1,8);
\draw [line width=1.2pt] (-1,8)-- (-1,7);
\draw [line width=1.2pt] (-7,2)-- (-5,2);
\draw [line width=1.2pt] (-6,3)-- (-6,1);
\draw [line width=1.2pt] (-4,7)-- (-4,3);
\draw [line width=1.2pt] (-5,4)-- (-3,4);
\draw [line width=1.2pt] (-5,5)-- (-3,5);
\draw [line width=1.2pt] (-5,6)-- (-3,6);
\draw [line width=1.2pt] (-2,8)-- (-2,7);
\draw (-7.58,1.24) node[anchor=north west] {$\mathbf{A}$};
\draw (-0.9,8.84) node[anchor=north west] {$\mathbf{B}$};
\end{tikzpicture}}
\end{figure}
\item (5\%) $(n!+2)!=(n!+2)(n!+1)n!$ para cualquier número natural $n$\dotfill{\bf \rule{1.5cm}{0.7pt}}
 
\end{enumerate}


{\bf PARTE II (80 \%):}\\
\textit{Resuelve en forma clara y ordenada cada uno de los problemas que se te 
presentan, dejando constancia de tus soluciones. Recordar que el examen es individual, es decir soluciones id\'enticas ser\'an anuladas. !`No se te olvide que est\'as 
en una academia de alto rendimiento!}


\begin{problema}
    ¿Cuántos números de $8$ dígitos hay, tales que son divisibles por $9$ y todos los dígitos son distintos?
\end{problema}

\begin{problema}
   ¿De cuántas maneras puedes colocar $5$ torres idénticas en un tablero de ajedrez de $6$ por $10$ de tal manera que ninguna de las torres se ataque entre sí? 
\end{problema}

\begin{problema}
   Utilizar el teorema del binomio para probar la fórmula:

   \begin{center}
       $\binom{n}{0} + \binom{n}{2} + \binom{n}{4} + ... = \binom{n}{1} + \binom{n}{3} + \binom{n}{5} + ...$
   \end{center}
\end{problema}

\newpage

\begin{problema}
   Demuestre por caminos la siguiente identidad

   \begin{center}
       $C^{n}_{4n} = C^{0}_{3n} C^{n}_{n} + C^{1}_{3n} C^{n-1}_{n} + C^{2}_{3n} C^{n-2}_{n} + ... + C^{n}_{3n} C^{0}_{n}$
   \end{center}
\end{problema}


%Consideren esta propuesta de crédito
\textbf{Crédito extra.} Considere todos los números de $7$ dígitos que contienen cada uno de los dígitos $1$, $2$, $3$, $4$, $5$, $6$, $7$ exactamente una vez y que no son divisibles por $5$. Están ordenados en una lista en orden creciente. Encuentra el número $2000$ en esta lista.


\end{document}