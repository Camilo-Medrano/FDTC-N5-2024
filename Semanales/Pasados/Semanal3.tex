\documentclass[12pt]{article}

%Paquetes a utilizarse
\usepackage[width=7in, height=9.5in, top=0.75in, papersize={8.5in,11in}]{geometry}
\usepackage[spanish]{babel} 
\decimalpoint
\usepackage[utf8]{inputenc}
\usepackage{bbding}
\usepackage[colorlinks = true, linkcolor = blue, urlcolor = BlueViolet, citecolor = OliveGreen]{hyperref}
\usepackage{graphicx}
\usepackage{amssymb,amsthm,amsmath}
\usepackage{enumerate}
\usepackage{array,multicol,multirow}
\usepackage{xcolor}
\usepackage{fancybox,tcolorbox}
\usepackage{caption,subcaption,float,tabularx}
\usepackage{enumitem}

\theoremstyle{definition}
\newtheorem{corolario}{Corolario}
\newtheorem{lema}[corolario]{Lema}
\newtheorem{proposicion}[corolario]{Proposición}
\newtheorem{teorema}[corolario]{Teorema}
\newtheorem{propiedad}[corolario]{Propiedad}
\newtheorem*{observacion}{Observación}
\newtheorem{definicion}{Definición}
\newtheorem*{demostracion}{Demostración}
\newtheorem{ejemplo}{Ejemplo}
\newtheorem{problema}{Problema}
\newtheorem*{solucion}{Solución}
\newtheorem{ejercicio}{\PencilRightDown \  Ejercicio}
\newtheorem{step}{Paso}
\newtheorem{credito}{Crédito}

\usepackage{tikz}
\usetikzlibrary{arrows.meta,babel,calc,positioning}

\renewcommand{\arraystretch}{1.5}
\providecommand{\abs}[1]{\lvert#1\rvert}
\providecommand{\norm}[1]{\lVert#1\rVert}

\renewcommand{\tabularxcolumn}[1]{m{#1}}
\newcommand{\Evaluacion}[4]{
\setcounter{ejercicio}{0}
\noindent\begin{tabular}{lcr}
	\includegraphics[height=3cm]{Logos/logo-UES.png}\hspace{2.5em}
	&
	\includegraphics[height=2.75cm]{Logos/logo-PJT.png}
	& 
	\hspace{2.5em}\includegraphics[height=2.75cm]{Logos/logo-MINEDUCYT.png}
\end{tabular}

\hfill

\begin{center}
    
    UNIVERSIDAD DE EL SALVADOR
    \\PROGRAMA JÓVENES TALENTO
    \\FDTC 2022
    \\#2
    \\Nivel Olímpico C de Matemáticas

\end{center}

\begin{center}
    #1
\end{center}

%\textbf{Nombre}: \enspace\hrulefill

#3

\input{#4}
\newpage
}

\newtheorem{obs}{Observación}

%\usepackage[margin=2.5cm]{geometry}
%\usepackage{wasysym}
%\usepackage{stmaryrd,textcomp}
%\usepackage{pgf,tikz}
%\usetikzlibrary{arrows}

\parskip = 2mm   %%%% genera un espacio de X mm entre lo párrafos
\parindent = 3mm
\usepackage{multicol}
\usepackage{iwona}

\newcommand{\tema}{Semana 3}
\newcommand{\fecha}{Sábado, 17 de diciembre de 2022}
\newcommand{\sesion}{Examen semanal}

\begin{document}
%\thispagestyle{empty}
%\newpage
\thispagestyle{empty}

\begin{figure}[h] 
	\begin{minipage}[b]{0.26\textwidth}
		\begin{center}
			\includegraphics[height=3cm]{Logos/UES.png}
			\par\end{center}
	\end{minipage} 
	\begin{minipage}[b]{0.46\textwidth}
		\begin{center}
			UNIVERSIDAD DE EL SALVADOR\\ [0.1cm]
			PROGRAMA JÓVENES TALENTO\\ [0.1cm]
	        FDTC 2022\\ [0.1cm]
                NIVEL 5\\ [0.1cm]
			COMBINATORIA 
			\par\end{center}
	\end{minipage} 
	\begin{minipage}[b]{0.05\textwidth}
		\begin{center}
			\includegraphics[height=2cm]{Logos/LOGO PJT.png}
			\par\end{center}
	\end{minipage}
\end{figure}

\begin{center}
    \begin{tabular}{p{4.5cm} p{7cm} p{4.5cm}}
        \tema & \centering\fecha & \hfill\sesion
    \end{tabular}
\end{center}



{\bf PARTE I (20 \%):}\\
\textit{Determine si las siguientes afirmaciones son falsas o verdaderas e indique con una $V$ en el caso de ser verdadera o una $F$ en el caso de ser falsa. Las afirmaciones no necesitan argumentación.}

  \begin{enumerate}

\item (5\%) Si $A$ y $B$ son conjuntos finitos de la misma cardinalidad, $|A\cup B| = 2022$ y $|A \cap B| = 22,$
entonces la cardinalidad de $A$ y $B$ es $1000$\dotfill{\bf \rule{1.5cm}{0.7pt}}
\item (5\%) El total de soluciones enteras no negativas de la ecuaci\'on $x_1+x_2+x_3=10$ es $\displaystyle C^3_9$\dotfill{\bf \rule{1.5cm}{0.7pt}}
\item (5\%) El número mínimo de estudiantes en un curso para asegurar que tres de ellos nacieron el mismo
mes es $23$\dotfill{\bf \rule{1.5cm}{0.7pt}}
\item (5\%) $5$ personas dejan sus abrigos en el guardaropas de un restaurante. Hay $44$ formas para que
ninguna persona reciba su sombrero.\dotfill{\bf \rule{1.5cm}{0.7pt}}
 
\end{enumerate}


{\bf PARTE II (80 \%):}\\
\textit{Resuelve en forma clara y ordenada cada uno de los problemas que se te 
presentan, dejando constancia de tus soluciones. Recordar que el examen es individual, es decir soluciones id\'enticas ser\'an anuladas. !`No se te olvide que est\'as 
en una academia de alto rendimiento!}


\begin{problema}
    Hallar el número de permutaciones de $1,2,3,4,5,6,7$ que no contengan a $3$ en el tercer lugar ni a $5$ en el quinto lugar.
\end{problema}

\begin{problema}
    Encuentra el número de formas en las que una empresa puede asignar $9$ proyectos a $4$ personas de modo que cada persona obtenga por lo menos un proyecto.
\end{problema}

\begin{problema}
    Hay $100$ personas sentadas en una mesa circular a distancia constante entre sí y al menos $51$ de ellas son mujeres. Verificar que hay al menos $2$ mujeres sentadas en posiciones diametralmente opuestas.
\end{problema}

\begin{problema}
    Demuestre que, dados $5$ puntos en el plano con coordenadas enteras, siempre hay dos tales que su punto medio tiene coordenadas enteras. Tenga en cuenta que las coordenadas del punto medio de dos puntos, $A=(x_1,y_1)$ y $B=(x_2,y_2)$, están dadas por $\displaystyle M=\left(\frac{x_1 + x_2}{2}, \frac{y_1 + y_2}{2}\right)$.
\end{problema}

\textbf{Crédito extra.} ¿Cuántas soluciones existen para la ecuación

\begin{center}
    $x_1+x_2+x_3 = 13$
\end{center}

si $5 \geq x_1 \geq 1$, $6 \geq x_2 \geq 1$ y $x_3 \geq 1$?

\end{document}